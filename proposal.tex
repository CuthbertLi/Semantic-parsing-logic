\documentclass[5pt]{article}
\usepackage[pdftex]{graphicx, color}
\usepackage{listings, proof}
\usepackage{tikz}
\usetikzlibrary{automata, positioning}
\usepackage{geometry, amsmath}

\newcommand{\infertext}[2]{\infer{{\textrm{#1}}}{#2}}

\begin{document}
Semantic Parsing Using Reinforcement Learning and Markov Logic

Probabilistic Grammars (Slide 17.15)

\section{Introduction} 
	Semantic parsing is the task of converting a natural language statement to a logical form: a machine-understandable representation of its meaning. 

	Formalization means representing natural languages in formal languages. In essence, the task of formalization is also a translation task. That is, translating a statement in a natural language into a formal language. 

	\subsection{Examples}
		First Example: 

		Input: For any set $X$ of nonempty sets, there exists a choice function $f$ defined on $X$. 

		Output: $\forall X (\emptyset \not\in X \Rightarrow \exists f: X \rightarrow \cup{X} \forall A \in X (f(A) \in A)) $

		Second Example: 

		Input: The rule for equality is that any types may be freely compared except Int, String and Bool, which may only be compared with objects of the same type. 

		Output: 

		\[
			\infertext{$e_1$ = $e_2$: Bool }{
				\begin{aligned}
				& \textrm{$O, M, C \vdash e_1: T_1$ }\\
				& \textrm{$O, M, C \vdash e_2: T_2$ }\\
				& \textrm{$T_1 \in$ \{Int, String, Bool\} $\vee$ $T_2 \in$ \{Int, String, Bool\} $\Rightarrow T_1 = T_2 $}
				\end{aligned}
			}
		\]

	\subsection{Motivating Applications}
		verification of softwares and computer security
		verification of programming languages
		checking correctness of mathematical proofs

\section{Related Works}
	Shi, Mihalcea (2004): an algorithm for open text shallow semantic parsing using a frame dataset (FrameNet) and a semantic network (WordNet)

	Poon, Domingos (2009): unsupervised semantic parsing

	Quirk, Mooney, Galley (2015): loosely synchronous systems perform best when learning semantic parsers for If-This-Then-That recipes

	Beltagy, Quirk (2016): improved semantic parsers for If-Then statements \textbf{(this technique is generic)}

	Guu, Pasupat, Liu, Liang (2017): learn a semantic parser using reinforcement learning and Maximum Marginal Likelihood

\section{Methodology (we may include some contents from slides)}
	Formal definition: Define context-free grammar or context-sensitive grammar for natural languages
	Reinforcement Learning (Model-based learning): Learn empirical MDP model for semantic parsing
	Unsupervised Machine Learning
	Supervised Machine Learning

\section{Input Models} 
	Chomsky Hierarchy: 
	https://en.wikipedia.org/wiki/Chomsky_hierarchy#The_hierarchy (Table)
	Probabilistic Grammars (Slide 17.15)

\section{Output Models}
	Markov Logic: more expressive, but more complicated (slide 11)
	First-order Logic: less expressive (slide 06)

\section{Extensions}
	We may use our result for tasks of style transfer 

	style transfer: transforming a sentence to alter a specific attribute (e.g., sentiment) while preserving its attribute-independent content. 

	ex: changing "screen is just the right size" to "screen is too small"

\end{document}
