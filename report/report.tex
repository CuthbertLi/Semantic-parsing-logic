\documentclass{article}
\usepackage[utf8]{inputenc}
\usepackage{listings}
\usepackage{xcolor}
\usepackage{graphicx, fancyhdr, amsmath, amssymb, amsthm, subfig}
\usepackage{indentfirst}
\usepackage{pdfpages}
\usepackage{dirtree}
\usepackage{lastpage, hyperref}
% \usepackage{hyperref, url}

% \pagestyle{fancy}
% \fancyhf{}
% \rhead{Page \thepage\ of \pageref{LastPage}}
\title{Semantic Parsing for First-order Logic}
\author{
	Li Dinghong\\
	SIST, Shanghaitech University\\
	27663262\\
	\{lidh1, liujd, wudi\}@shanghaitech.edu.cn
}

\begin{document}
% abstract and beginnings
{
	\newpage
	\maketitle

	\textbf{Abstract}-{In this paper, we present a generic model for semantic parsing. }

	\vspace{5pt}
	\textbf{\emph{Keywords:}} {}

	\tableofcontents
}

\section{Introduction}{
	\subsection{Background}{
		Goal of semantic parsing is to formalize natural languages. That is, translating texts in natural languages into formal languges, like first-order logic (FOL) or higher-order logic. 
	}

	\subsection{Motivating Applications}{
		\cite{su} 
	}

	\subsection{Our Contributions}{}
}

\section{Methods}{
	\subsection{Recursive Definition of FOL}{}

	\subsection{Parsing Tree}{}
}

\section{Two-step Semantic Parsing}{
	\subsection{Lexical Analysis and Syntax Analysis}{}

	\subsection{Parsing Tree Traversal}{}
}

\section{Generation of FOL}{}

\section{Results}{}

\section{Conclusion}{}

\bibliographystyle{unsrt}
\bibliography{report}
\citation

\end{document}