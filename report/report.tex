\documentclass{article}
\usepackage[utf8]{inputenc}
\usepackage{listings}
\usepackage{xcolor}
\usepackage{graphicx, fancyhdr, amsmath, amssymb, amsthm, subfig}
\usepackage{indentfirst}
\usepackage{pdfpages}
\usepackage{dirtree}
\usepackage{lastpage, hyperref}
% \usepackage{hyperref, url}

% \pagestyle{fancy}
% \fancyhf{}
% \rhead{Page \thepage\ of \pageref{LastPage}}
\title{Semantic Parsing for First-order Logic}
\author{
	Li Dinghong\\
	SIST, Shanghaitech University\\
	27663262\\
	\{lidh1, liujd, wudi\}@shanghaitech.edu.cn
}

\begin{document}
% abstract and beginnings
{
	\newpage
	\maketitle

	\textbf{Abstract}-{}

	\vspace{5pt}
	\textbf{\emph{Keywords:}} {}

	\tableofcontents
}

\section{Introduction}{
	Goal of semantic parsing is to formalize natural languages. That is, translating texts in natural languages into formal languges, like first-order logic (FOL) or higher-order logic. 

	\subsection{Motivating Applications}{
		\cite{su} 
	}
	
	\subsection{Related Works and Our Contributions}{
		In this work, we choose to use English as input of our semantic parser, and FOL as output. 

		% For text generator, we use FOL as our input, and both English and Chinese as our output. 

		% Our model is highly generic. It can be easily extended to other natural languages, since it takes parsing tree as input, instead of original texts. 

		Thanks to parsing trees, our model has become much more generic. We take parsing tree as input, instead of directly using original texts. Thus, our model can easily be extended to other natural languages like French by defining new lexicon and grammar. 

		We succeed to include predicate logic in our output. It improves expressiveness of our models. 

		% As for text generation, our models 
	}
}

\section{Methods}{
	\subsection{Recursive Definition of FOL}{}

	\subsection{Context-free Grammar and Dependency Grammar}{}
}

\section{Two-step Semantic Parsing}{
	\subsection{Lexical Analysis and Syntax Analysis}{}

	\subsection{Parsing Tree Traversal}{}
}

\section{Generation of FOL}{}

\section{Results}{}

\section{Conclusion}{}

\bibliographystyle{unsrt}
\bibliography{report}
\citation

\end{document}